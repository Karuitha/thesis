%%%%%%%%%%%%%%%%%%%%%%%%%%%%%%%%%%%%%%%%%%%%%%%%%%%%%%%%%%%%%%%
%% BRIEF VERSION OF OXFORD THESIS TEMPLATE FOR CHAPTER PREVIEWS

%%%%% CHOOSE PAGE LAYOUT
% format for PDF output (ie equal margins, no extra blank pages):
\documentclass[a4paper,nobind]{templates/ociamthesis}

% UL 5 January 2021 - add packages used by kableExtra
\usepackage{booktabs}
\usepackage{longtable}
\usepackage{array}
\usepackage{multirow}
\usepackage{wrapfig}
\usepackage{colortbl}
\usepackage{pdflscape}
\usepackage{tabu}
\usepackage{threeparttable}
\usepackage{threeparttablex}
\usepackage[normalem]{ulem}
\usepackage{makecell}
\usepackage[colorlinks=false,pdfpagelabels,hidelinks=]{hyperref}
\usepackage{float}


%UL set section header spacing
\usepackage{titlesec}
% 
\titlespacing\subsubsection{0pt}{24pt plus 4pt minus 2pt}{0pt plus 2pt minus 2pt}

% UL 30 Nov 2018 pandoc puts lists in 'tightlist' command when no space between bullet points in Rmd file
\providecommand{\tightlist}{%
  \setlength{\itemsep}{0pt}\setlength{\parskip}{0pt}}
 
% UL 1 Dec 2018, fix to include code in shaded environments
\usepackage{color}
\usepackage{fancyvrb}
\newcommand{\VerbBar}{|}
\newcommand{\VERB}{\Verb[commandchars=\\\{\}]}
\DefineVerbatimEnvironment{Highlighting}{Verbatim}{commandchars=\\\{\}}
% Add ',fontsize=\small' for more characters per line
\usepackage{framed}
\definecolor{shadecolor}{RGB}{248,248,248}
\newenvironment{Shaded}{\begin{snugshade}}{\end{snugshade}}
\newcommand{\AlertTok}[1]{\textcolor[rgb]{0.94,0.16,0.16}{#1}}
\newcommand{\AnnotationTok}[1]{\textcolor[rgb]{0.56,0.35,0.01}{\textbf{\textit{#1}}}}
\newcommand{\AttributeTok}[1]{\textcolor[rgb]{0.77,0.63,0.00}{#1}}
\newcommand{\BaseNTok}[1]{\textcolor[rgb]{0.00,0.00,0.81}{#1}}
\newcommand{\BuiltInTok}[1]{#1}
\newcommand{\CharTok}[1]{\textcolor[rgb]{0.31,0.60,0.02}{#1}}
\newcommand{\CommentTok}[1]{\textcolor[rgb]{0.56,0.35,0.01}{\textit{#1}}}
\newcommand{\CommentVarTok}[1]{\textcolor[rgb]{0.56,0.35,0.01}{\textbf{\textit{#1}}}}
\newcommand{\ConstantTok}[1]{\textcolor[rgb]{0.00,0.00,0.00}{#1}}
\newcommand{\ControlFlowTok}[1]{\textcolor[rgb]{0.13,0.29,0.53}{\textbf{#1}}}
\newcommand{\DataTypeTok}[1]{\textcolor[rgb]{0.13,0.29,0.53}{#1}}
\newcommand{\DecValTok}[1]{\textcolor[rgb]{0.00,0.00,0.81}{#1}}
\newcommand{\DocumentationTok}[1]{\textcolor[rgb]{0.56,0.35,0.01}{\textbf{\textit{#1}}}}
\newcommand{\ErrorTok}[1]{\textcolor[rgb]{0.64,0.00,0.00}{\textbf{#1}}}
\newcommand{\ExtensionTok}[1]{#1}
\newcommand{\FloatTok}[1]{\textcolor[rgb]{0.00,0.00,0.81}{#1}}
\newcommand{\FunctionTok}[1]{\textcolor[rgb]{0.00,0.00,0.00}{#1}}
\newcommand{\ImportTok}[1]{#1}
\newcommand{\InformationTok}[1]{\textcolor[rgb]{0.56,0.35,0.01}{\textbf{\textit{#1}}}}
\newcommand{\KeywordTok}[1]{\textcolor[rgb]{0.13,0.29,0.53}{\textbf{#1}}}
\newcommand{\NormalTok}[1]{#1}
\newcommand{\OperatorTok}[1]{\textcolor[rgb]{0.81,0.36,0.00}{\textbf{#1}}}
\newcommand{\OtherTok}[1]{\textcolor[rgb]{0.56,0.35,0.01}{#1}}
\newcommand{\PreprocessorTok}[1]{\textcolor[rgb]{0.56,0.35,0.01}{\textit{#1}}}
\newcommand{\RegionMarkerTok}[1]{#1}
\newcommand{\SpecialCharTok}[1]{\textcolor[rgb]{0.00,0.00,0.00}{#1}}
\newcommand{\SpecialStringTok}[1]{\textcolor[rgb]{0.31,0.60,0.02}{#1}}
\newcommand{\StringTok}[1]{\textcolor[rgb]{0.31,0.60,0.02}{#1}}
\newcommand{\VariableTok}[1]{\textcolor[rgb]{0.00,0.00,0.00}{#1}}
\newcommand{\VerbatimStringTok}[1]{\textcolor[rgb]{0.31,0.60,0.02}{#1}}
\newcommand{\WarningTok}[1]{\textcolor[rgb]{0.56,0.35,0.01}{\textbf{\textit{#1}}}}

%UL 2 Dec 2018 add a bit of white space before and after code blocks
\renewenvironment{Shaded}
{
  \vspace{10pt}%
  \begin{snugshade}%
}{%
  \end{snugshade}%
  \vspace{8pt}%
}
%UL 2 Dec 2018 reduce whitespace around verbatim environments
\usepackage{etoolbox}
\makeatletter
\preto{\@verbatim}{\topsep=0pt \partopsep=0pt }
\makeatother

%UL 28 Mar 2019, enable strikethrough
\usepackage[normalem]{ulem}

%UL use soul package for correction highlighting
\usepackage{soul}
\usepackage{xcolor}
\newcommand{\ctext}[3][RGB]{%
  \begingroup
  \definecolor{hlcolor}{#1}{#2}\sethlcolor{hlcolor}%
  \hl{#3}%
  \endgroup
}
\soulregister\ref7
\soulregister\cite7
\soulregister\autocite7
\soulregister\textcite7
\soulregister\pageref7

%UL 3 Nov 2019, avoid mysterious error from not having hyperref included
\usepackage{hyperref}

%%%%% SELECT YOUR DRAFT OPTIONS
% Three options going on here; use in any combination.  But remember to turn the first two off before
% generating a PDF to send to the printer!

% This adds a "DRAFT" footer to every normal page.  (The first page of each chapter is not a "normal" page.)

% This highlights (in blue) corrections marked with (for words) \mccorrect{blah} or (for whole
% paragraphs) \begin{mccorrection} . . . \end{mccorrection}.  This can be useful for sending a PDF of
% your corrected thesis to your examiners for review.  Turn it off, and the blue disappears.

%%%%% BIBLIOGRAPHY SETUP
% Note that your bibliography will require some tweaking depending on your department, preferred format, etc.
% The options included below are just very basic "sciencey" and "humanitiesey" options to get started.
% If you've not used LaTeX before, I recommend reading a little about biblatex/biber and getting started with it.
% If you're already a LaTeX pro and are used to natbib or something, modify as necessary.
% Either way, you'll have to choose and configure an appropriate bibliography format...

% The science-type option: numerical in-text citation with references in order of appearance.
% \usepackage[style=numeric-comp, sorting=none, backend=biber, doi=false, isbn=false]{biblatex}
% \newcommand*{\bibtitle}{References}

% The humanities-type option: author-year in-text citation with an alphabetical works cited.
% \usepackage[style=authoryear, sorting=nyt, backend=biber, maxcitenames=2, useprefix, doi=false, isbn=false]{biblatex}
% \newcommand*{\bibtitle}{Works Cited}

%UL 3 Dec 2018: set this from YAML in index.Rmd
\usepackage[style=numeric-comp, sorting=none, backend=biber, doi=false, isbn=false]{biblatex}
\newcommand*{\bibtitle}{References}

% This makes the bibliography left-aligned (not 'justified') and slightly smaller font.
\renewcommand*{\bibfont}{\raggedright\small}

% Change this to the name of your .bib file (usually exported from a citation manager like Zotero or EndNote).
\addbibresource{references.bib}

%%%%% YOUR OWN PERSONAL MACROS
% This is a good place to dump your own LaTeX macros as they come up.

% To make text superscripts shortcuts
	\renewcommand{\th}{\textsuperscript{th}} % ex: I won 4\th place
	\newcommand{\nd}{\textsuperscript{nd}}
	\renewcommand{\st}{\textsuperscript{st}}
	\newcommand{\rd}{\textsuperscript{rd}}

%%%%% THE ACTUAL DOCUMENT STARTS HERE
\begin{document}

%%%%% CHOOSE YOUR LINE SPACING HERE
% This is the official option.  Use it for your submission copy and library copy:
\setlength{\textbaselineskip}{22pt plus2pt}
% This is closer spacing (about 1.5-spaced) that you might prefer for your personal copies:
%\setlength{\textbaselineskip}{18pt plus2pt minus1pt}

% UL: You can set the general paragraph spacing here - I've set it to 2pt (was 0) so
% it's less claustrophobic
\setlength{\parskip}{2pt plus 1pt}

% Leave this line alone; it gets things started for the real document.
\setlength{\baselineskip}{\textbaselineskip}

% all your chapters and appendices will appear here
\hypertarget{tables}{%
\chapter{Possible Impacts of the Transformation of MFIS: A Conceptual Framework}\label{tables}}

\chaptermark{Conceptual Framework on Transformation of Microfinance Institutions}

\minitoc 

\hypertarget{introduction}{%
\section{Introduction}\label{introduction}}

\noindent This section focuses on outlining a conceptual framework of the possible impacts of the transformation of MFIs globally and particularly in Africa. The chapter first addresses the low level of transformation of MFIs across the globe. Next, the section describes the theoretical basis for the link between MFI transformation and performance, followed by the techniques for measuring the financial and social performance of MFIs. Subsequently, the section provides a critical review of the empirical literature on the effects of the transformation of MFIs, and the relationship between ownership structure, organisational structure, and efficiency of the transformed MFIs, on the one hand, and their financial and social performance, on the other. The section concludes by examining the determinants of the performance of transformed MFIs.

\hypertarget{drivers-roadblocks-to-mfi-transformation}{%
\section{Drivers/ Roadblocks to MFI Transformation}\label{drivers-roadblocks-to-mfi-transformation}}

\noindent As noted by \textcite{d2017ngos}, although a substantial number of MFIs have adopted the commercial model, a significant number still operate as NGOs. Given the benefits that come with commercialisation, it is hard to justify the low levels of transformation among MFIs. Further, MFIs seeking to transform usually have substantial access to financial and technical support before and after transformation \autocite{bateman2010doesn,campion1999institutional} from organisations like the World Bank, and USAID. If the transformation of MFIs is so desirable, what is hindering the mass transformation of MFIs to the commercial model? Put another way, what factors drive, enable or hinder the transformation of MFIs? The section highlights the challenges that MFIs face in the transformation process and draws insights from institutional theorists.

\hypertarget{financial-and-administrative-challenges}{%
\subsection{Financial and Administrative Challenges}\label{financial-and-administrative-challenges}}

\noindent The costs and administrative challenges of the transformation could partly explain why so many of the MFIs continue to operate as NGOs. \textcite{campion1999institutional} note, an MFI should only undertake to transform when it is internally capable of bearing the costs and time involved. For instance, an MFI should transform when it can navigate the administrative pitfalls, like the process of applying for licensing and the attendant requirements. It is not clear at what point an MFI is internally capable of transforming either regarding financial resources (assets base) or length of operation or a combination of these and other factors. Campion and White further advised that MFIs should consider transformation only when it is the best strategy, and the regulatory environment is conducive. Researchers have not addressed these issues comprehensively in the existing literature to pinpoint the suitable preconditions for transformation. Alternate explanations for the failure by MFIs to transform come from the institutional theory.

\hypertarget{institutional-theory-perspective}{%
\subsection{Institutional Theory Perspective}\label{institutional-theory-perspective}}

\noindent The institutional theory gives some useful insights regarding inertia on the transformation of MFIs. The institutional theory seeks to explain the development of the structure of rules, norms, and routines in organisations. The theory examines how specific structures get accepted as the standard ways guiding social behaviour, and ultimately decline and get discarded \autocite{scott2005institutional}. The theory holds that the institutional environment is more influential in the development of formal structures in organisations that market pressures \autocite{powell2012new}.

The institutional theory is not only focused on the persistence and convergence in organisations but is also concerned with change and de-institutionalisation. The institutional theory aims to explain the creation, transformation\footnote{In the case of MFIs, the transformation involves the deinstitutionalization of the NGO model and the institutionalization and legitimation of the commercial model of conducting MF business.} and the impacts of institutions\autocite{powell2012new}. Thus, although economic pressures have a bearing on organisations, the institutional environment significantly influences the way that actors interpret the meaning and consequences of the economic forces. Finally, the institutional theory notes that organisations often adopt established institutional structures without critical scrutiny on the path to gaining legitimacy in the institutional environment \autocite{tina2002institutional}.

The institutional theory draws from a variety of disciplines. In the 19th century, the theory borrowed heavily from economics, political science, and sociology. The works of Karl Marx, Emile Durkheim, Marx Weber, Herbert Spencer and Charles Cooley were particularly influential. Later, the theory expanded and borrowed from cognitive psychology, evolutionary and transaction cost economics, social history, ethnomethodology, and the new cultural studies in anthropology and sociology. Economists such as John Commons and Thorstein Veblen had argued for the examination of the historically changing rules that govern economic transactions. They pointed out that economics had overemphasised rationality and ignored habit, routines, procedures, conventions, roles, and rules as drivers of economic behaviour \autocite{scott2005institutional}. Moreover, historical institutionalism stresses the role which current institutional structures play to influence subsequent arrangements and directions of change. Accordingly, historical institutionalists view the evolution of institutional structures as path dependent \autocite{tina2002institutional}.

As detailed by \textcite{ritzer2004encyclopedia}, institutional theory rests on a set of assumptions. The first assumption is that institutions, as governance structures, embody rules for social conduct. Secondly, actors accord legitimacy to those groups and organisations that conform to these rules. This legitimation is a condition that contributes to the survival of firms. In the case of MFIs, offering financial services to the poor and the financially excluded is the source of legitimacy which some researchers argue is under threat due to the transformation. Moreover, the theory assumes that inertia, a tendency to resist change, is inherent in institutions. Lastly, the historical evolution of institutions matters. The emphasis on the historical setting implies that past institutional structures constrain and channel new arrangements and shapes future evolution of firms.

One form of pressure from the institutional environment is coercive pressure. Coercion implies some form of implicit or explicit push to conform. Coercive pressure is exceptionally high where the institution is highly dependent on the institutional environment from which the pressure to conform arises \autocite{meyer1977institutionalized}. Pioneer MFIs were relatively more donor-dependent. Hence, donors like the World Bank and other donor agencies, had leverage to place considerable pressure on MFIs to transform especially with the rise of neo-liberalism \autocite{ostry2016neoliberalism}. In the case of the institutional transformation of MFIs, the donors theorised against the existing, not-for-profit organisational model. The theorisation included the specification of the failings of the not-for-profit model and the vindication of the practices of the commercial approach in light of pragmatic considerations \autocite{tina2002institutional} and real or perceived benefits.

The institutional theory not only provides insights regarding the transformation of MFIs but also insights into the financial and social performance of MFIs post-transformation. \textcite{thornton2002rise} provides an analogy of publishing firms that transformed from the professional editorial setup to the market-led positioning, a trend prevalent in numerous industries \autocite{thornton2015institutional}. The author argues as follows, in line with the institutional theory:

\begin{quote}
In the context of the publishing industry, the meaning and legitimacy of various sources of organisational identity and strategy and structure are shaped by a prevailing institutional logic. Publishers may identify with publishing as a profession by building their personal reputations in the industry, or publishers may identify with publishing as a business by improving the market positions of their firms. Second, institutional logics determine which issues and problems are salient and the focus of management's attention. For example, publishers may focus on increasing sales by concentrating on author-editor networks in product development; or publishers may focus on increasing profits by emphasising control of resource competition in the product-market. Third, institutional logics determine which answers and solutions are the focus of management's attention. For example, publishers may adopt strategies of growth by focusing on organic growth and building personal imprints; or publishers may concentrate on acquisition growth and building market channels\autocite{thornton2002rise}.
\end{quote}

The dilemmas that publishers faced is similar to what MFIs are facing currently. In the context of transformed MFIs, the manifestation of social conduct is the ability to meet their social mission by focusing on the financially excluded. On the other hand, MFIs are expected to meet a new threshold of financial sustainability. The research debates points to the rationalisation of the new MFI business model. However, failure to conform to the expectations of attaining their social mission may render the new business model and even the entire MF industry illegitimate. The lack of legitimacy may put the survival of the transformed MFIs at risk.

Although the transformation of some MFIs fits well within the explanations of the institutional theory, it fails to account for many MFIs that have not transformed. The critics of the institutional theory, however, provide some useful insights into the persistence of non-transformation of MFIs. The critics contend that the theory is used merely to explain the homogeneity and persistence of phenomena and that it does not explain ``de-institutionalisation'' equally well. They argue that organisations do not merely drive change, but that they also change over time, often in response to external demands or pressure. Furthermore, the rate and nature of change differ across organisations \autocite{tina2002institutional}.

Contemporary research lumps together all MFIs and attempts to describe aggregate changes. However, given the specificity of the drivers to transform, such studies do not give a proper account of the reasons for transformation or the lack of it. Similarly, analysis based on global level data may crowd out institutional specific and regional specific manifestations of the institutional transformation of MFIs. It would be worthwhile to perform an in-depth study of individual organisations to uncover individual peculiarities because of the changes at the micro level, which could also include institutional specific, country-specific and regional specific drivers for MFI transformation \autocite{mersland2009performance}.

Little empirical research has explored the factors that drive the decision by MFIs to transform. Most of the researchers take transformation for granted, or maybe as an occurrence that unfolds as the MFI matures. It would be worthwhile to establish the institutional and other factors that hinder the transformation of MFIs to commercial entities. Such research could point out the characteristics of MFIs that have undergone a successful transformation. Also, the research would link the pre-transformation state of the MFIs and the post-transformation financial and social performance. The output from such specifically oriented research would then point to the indicators that the management of MFIs and policymakers would consider when evaluating the success or failure of a potential MFI transformation.

\hypertarget{mfi-transformation-and-performance-theoretical-insights}{%
\section{MFI Transformation and Performance: Theoretical Insights}\label{mfi-transformation-and-performance-theoretical-insights}}

\noindent Researchers have widely explored the theme of the transformation of MFIs. The primary literature, for example,\autocite{campion1999institutional,mersland2010microfinance} mainly dealt with the implications of the transformation of MFIs, especially regarding possible mission drift. Other studies have compared commercial MFIs against those MFIs operating as NGOs. Some researchers have welcomed the presence of NGOs in the MF industry as an attempt to address market failures. However, some scholars have attacked the comparison between NGO and commercial MFIs. Central to their argument is that utility maximisation, not profit maximisation better explains the behaviour of firms in markets with mixed preferences \autocite{bos2015practice}. This section provides a theoretical background on the link between the transformation of MFIs and their performance.

\hypertarget{agency-theory}{%
\subsection{Agency Theory}\label{agency-theory}}

\noindent This section describes the agency theory and its implication to the study.

\hypertarget{what-is-agency-theory}{%
\subsubsection{What is Agency Theory?}\label{what-is-agency-theory}}

\noindent Agency theory deals with the principal-agent relationships where one party (the principal) engages another party (the agent) to act on their behalf. The principal delegate's decision making to the latter. Both the principal and the agent are interdependent but may have different goals and attitudes towards risk \autocite{ballwieser2012agency,eisenhardt1989agency}. Suffice it to note that the principal-agent relationships extend beyond that between shareholders and managers, and could include, among others, the relationship between shareholders and debtholders.

\hypertarget{historical-overview-of-the-agency-theory}{%
\subsubsection{Historical Overview of the Agency Theory}\label{historical-overview-of-the-agency-theory}}

\noindent Although hardly acknowledged, Adam Smith had mentioned the concept of agency theory arising out of the separation of ownership from management in 1776. In line with the agency theory, Smith stated the following in his renowned book, \emph{The Wealth of Nations}.

\begin{quote}
The directors of such (joint-stock) companies, however, being the managers rather of other people's money than of their own, it cannot well be expected, that they should watch over it with the same anxious vigilance with which the partners in a private copartnery frequently watch over their own. Like the stewards of a rich man, they are apt to consider attention to small matters as not for their master's honour and very easily give themselves a dispensation from having it. Negligence and profusion, therefore, must always prevail, more or Iess, in the management of the affairs of such a company- Adam Smith \autocite{jensen1976theory}.
\end{quote}

In an agency relationship, information asymmetry makes it difficult or expensive to verify what the agent is doing. Also, the divergence in attitude towards risk between the two parties is problematic because the objectives of both the principal and the agent may differ based on their risk preferences \autocite{sun2016ownership}. Ex-ante, agents may misrepresent their abilities, and the principal cannot verify their skills at the time of hiring or even when the agent is working \autocite{eisenhardt1989agency}. The unverifiability of competencies presents the adverse selection problem. On the other hand, the agents may, after hiring, exert too little effort in their work, and also undertake other activities that are harmful to the principal \autocite{tirole2010theory}. The ex-post manifestation of information asymmetry is the moral hazard problem.

Following Smith, \textcite{berle1991modern} (originally published in 1932) confirmed the extent of the dispersion of ownership arising from the separation of ownership from control in industrial firms in the United States. They traced the separation to the industrial revolution, the rise of stock markets, and the advent of organised labour and trade unions \autocite{bendickson2016agency}. Berle and Means noted that the separation meant that the suppliers of capital essentially gave up ownership to become mere recipients of financial returns, turning private corporations into quasi-public organisations. They argued that such separation disenfranchised the owners by ceding too much power to managers. Consequently, as subsequent research indicated, the dispersed ownership would result in shareholder apathy\footnote{The apathy arises due to the free rider problem. Where ownership is dispersed, no individual stockholder will be motivated to monitor the management because the entire cost of monitoring is borne by the individual shareholder, whereas the benefits accrue to all. This gives rise to rational apathy that allows the management to run the firm at their whims.} in monitoring the management \autocite{battaglini2016dynamic}.

With the inevitable separation of shareholding and management, monitoring by institutional and individual investors holding large blocks of shares could mitigate shareholder apathy. Large investors have the capacity and incentive to monitor the management given that the benefits they derive from monitoring exceed the costs of failing to do so. However, the shareholders holding a substantial equity stake could, on the other hand, abuse their position by extracting private benefits of control \autocite{dyck2004private,doidge2004us}. Hence, there is a trade-off between the potential advantages in the concentration of ownership and the drawbacks of near-absolute dispersion. Considering their findings, Berle and Means asserted that a corporate revolution had occurred and predicted that corporations characterised by dispersed ownership would dominate economic activities. They also projected growth in the size of firms and the concentration of the economy.

\hypertarget{positivist-versus-the-principal-agent-approach-to-agency-theory}{%
\subsubsection{Positivist Versus the Principal-Agent Approach to Agency Theory}\label{positivist-versus-the-principal-agent-approach-to-agency-theory}}

\textcite{genoe2011james} (published in 1941) stated that managers are self-interested and opportunistic, assumptions that later became central to the agency theory \autocite{bendickson2016agency}. Subsequently, research on agency theory followed two strands: the positivist approach and the classic principal-agent approach. As \textcite{eisenhardt1989agency} notes, the two methods share a standard unit of analysis and assumptions but differ in the application of mathematical techniques, the choice of the dependent variable, and style.

The positivist approach focused on identifying the situations where the objectives of the owners and the managers diverge and suggested mechanisms for aligning the objectives. In doing so, the positivists dwelt exclusively on the agency conflicts that arise between the owners (shareholders) and the managers. Prominent examples of positivist research include \autocite{jensen1976theory,fama1980agency,jensen1983organization}. \textcite{jensen1976theory} examined ownership structure, including how equity ownership by managers could mitigate agency conflicts. \textcite{fama1980agency} considered the role of capital and labour markets as information mechanisms to control the management. \textcite{fama1983agency} posited that the board of directors was an information system that could be applied to lessen opportunism by managers in large corporations. The positivist approach is the most prominent in the finance literature and forms the basis of the various standard definitions of the agency theory. The paper by \textcite{jensen1976theory} is especially significant.

In this influential paper, \textcite{jensen1976theory} noted that the separation of ownership from the day to day management of the firm gave rise to agency conflicts due to moral hazards where managers do not always act in the best interests of the shareholders. These agency conflicts manifest themselves in the form of insufficient effort, entrenchment strategies, extravagant investments, and self-dealing by the management \autocite{tirole2010theory}.

Researchers have proposed mechanisms to resolve the agency conflicts, and as a result, incur agency costs. The first is the use of incentives to design an efficient principal-agent risk bearing mechanism, for example, by the use of performance-based compensation schemes. Secondly, the principal must monitor the performance of managers- for instance through the employment of auditors. Alternatively, the managers incur bonding costs to demonstrate a commitment to the principal's goals \autocite{bosse2016agency}.

In markets with weak investors protection regulations and an extreme dispersion in ownership, institutional investors and large individual investors holding controlling stakes are better placed to monitor managers \autocite{goergen2003levels}. Thus, the ownership of corporations has implications on the level of monitoring and presumably the performance of firms. Finally, other mechanisms aimed at aligning the interests of the principal with those of the agents including ceding some proportion of ownership to the managers \autocite{ang2000agency}, the managerial labour market and the market for corporate control \autocite{ballwieser2012agency}. However, it is not possible to design a scheme that will fully align the incentives of managers and shareholders. This divergence results in residual loss \autocite{fama1983agency}.

However, agency conflicts do arise between other parties beyond owners and managers. In this respect, the classic principal-agent approach extended the application of agency theory to the other forms of agency relationships such as shareholders-debtholders, employer-employee, lawyer-client, and buyer-supplier. Logical deduction and rigorous mathematical proofs are the hallmarks of the principal-agent approach. The output from this school is a set of general theories applicable to a broad variety of principal-agent relationships. Principal-agent literature addresses the agency relationship between shareholders and bondholders.

The prominent principal-agent literature includes \textcite{demski1978economic}, \textcite{eisenhardt1985control}, and \textcite{eisenhardt1988agency}. In brief, the paradigm prescribes the optimal contract (be it outcome-based or behaviour based) between the principal and the agent under varying degrees of outcome uncertainty, risk aversion, information systems, goal conflict, task programmability, outcome measurability, and the length of the agency relationship \autocite{eisenhardt1989agency}. Agency theory: An assessment and review. C. has presented a comprehensive overview of both the positivist and the principal-agent view of the principal-agency relationship.{]}. Both the positivist and the principal-agent paradigms are, however, complementary. The positivist approach is a subset of the broader principal-agent approach. While the positivist approach identifies the different contract alternatives to mitigate the agency conflicts, the principal-agent approach prescribes the most efficient contractual arrangement.

\hypertarget{criticism-of-the-agency-theory}{%
\subsubsection{Criticism of the Agency Theory}\label{criticism-of-the-agency-theory}}

Nevertheless, several researchers have criticised the agency theory. Much of the criticism has been directed mainly at the more popular positivist paradigm, with little of it aimed at the principal-agent paradigm. Some researchers hail the agency theory for offering a more complex view of organisations. However, \textcite{perrow1986complex} condemned the positivist agency theory for being ``trivial, dehumanising, and even dangerous'' \textcite{eisenhardt1989agency} while others scholars view it as minimalist \autocite{hirsch1986collaboration}, tautological and lacking rigour \autocite{jensen1983organization}. \textcite{perrow1986complex} further asserted that the theory addressed no apparent problems and was one-sided as it ignored the potential exploitation of workers. Also, he questioned the empirical verifiability of the theory, although subsequent research did test the theory empirically \autocite{eisenhardt1989agency}. The view by Perrow is in line with \textcite{hirsch1986collaboration} who assert that the agency theory is excessively narrow, and focused only on the stock price.

Although empirically verifiable, agency theory fares weakly in such tests. For example, \textcite{jensen1990performance} found no conclusive link between CEO pay and stock price. \textcite{frydman2010ceo} found weak evidence for incentive alignment as an explanatory agency construct for CEO pay. \textcite{tosi2000much}, after reviewing US executive compensation from 1936 to 2005 also found that agency theory is inconsistent with the available evidence. The most stinging rebuke of the agency theory came after the global financial crisis that started in 2007. Research into the causes of the crisis uncovered that strong incentives are not always optimal. The sub-optimality of strong incentives is especially prominent where there are no precise measures of the agent's efforts, in situations that require multitasking, and where cooperation between agents is essential. In such circumstances,\textcite{roberts2010designing} argues that weak rather than strong incentives are necessary.

Additional criticism of the agency theory arises from assumptions. For instance, by self-interest, agency theory assumes that the providers of capital are responsible, whereas, the managers are opportunists waiting for every chance to swindle. However, the same could also hold for the providers of capital. As \textcite{miller2011angel} illustrated in their case study, the providers of capital may act against the long-term interest of the firm by forcing managers into actions that only have short-term benefits, but that is harmful in the long term. On the other hand, some steward like agents may use their superior information for the long-term benefit of stakeholders. Such short-termism has been the subject of research since the global financial crisis of 2007, yet there is scant research on the drivers of the myopic management \autocite{dallas2011short}.

Furthermore, in criticising the assumption of self-interest, \textcite{bosse2016agency} view self-interest from the context of fairness and reciprocity on the part of the CEO. They argue that in the implementation of rules to check CEO discretion, boards must act with positive reciprocity to minimise welfare reducing revenge acts from the CEO. Also, the persistence of CSR initiatives by firms may be indicative of some degree of altruism amongst investors, although the motives of CSR are subject to debate \textcite{glegg2018corporate} which could be present in the MFIs even after the transformation. Nevertheless, research indicates that commercial providers of funds are driven by commercial logic, while development logic drives donors \autocite{cobb2016funding}. The self-interest assumption implies that with commercialisation, the social dimension of microfinance is likely to be secondary to the achievement of financial objectives.

Furthermore, the assumptions that organisations are profit-seeking is in direct conflict with the operation of MFIs pre-transformation. Evidence on transformed MFIs however, indicates a shift towards profit positioning \autocite{chahine2010social,d2017ngos}. Also, agency theory leaves no room for non-pecuniary, intrinsic motivation, which is subject to debate. Furthermore, agency theory presumes that the agent's utility is positively contingent on financial incentives, and negatively with effort. Moreover, the theory assumes that time preferences are calculated mathematically using the exponential discount function. Finally, the theory assumes that both effort and motivation increase monotonically with rewards. All these assumptions have come under criticism and led to the development of behavioural agency theory\^{}{[}The behavioural agency theory relaxes some of the assumptions of the mainstream agency theory. In contrast with the standard agency theory which focuses on monitoring costs and incentive alignment, the behavioural agency theory (BAT) focuses on agent performance. The BAT argues that the objectives of the shareholders are likely to be achieved if the agent is motivated to perform to the best of their ability. The theory incorporates intrinsic motivation and inequity aversion, and relaxes assumptions on risk and time discounting. A detailed presentation of the study is presented by \autocite{pepper2015behavioral}.

The other limitations of the agency theory usually arise out of the assumptions that firms operate in a tax-free environment and that stocks are non-voting. While these assumptions allow for the development of the model, they are unrealistic. The existence of taxes may explain why firms take up debt, but this does not explain why debt structures vary across industries and companies, and over time \autocite{deangelo2015stable}, although differentials in the magnitude of the initial capital outlay explain part of the cross-industry difference.

Finally, the theory fares poorly in explaining the governance and performance of firms that are family-owned, firms engaged in strategic entrepreneurship, and those that take part in social entrepreneurship that have a double bottom line {[}bendickson2016agency{]} MFIs fall into the latter category, given that they have both a financial goal and a social mission. As \textcite{bendickson2016agency} conclude, insights from network theory and the social exchange theory could when used together with the agency theory, better explain the performance of firms characterised by the double bottom line.

\hypertarget{implications-of-agency-theory-of-mfi-transformation}{%
\subsubsection{Implications of Agency Theory of MFI Transformation}\label{implications-of-agency-theory-of-mfi-transformation}}

\noindent Given the agency resolution mechanisms, the transformation of MFIs to commercial entities has several implications. First, the entry of shareholders and debt-holders can induce discipline among managers given the increased monitoring. Thus, commercialisation could influence financial performance and the efficiency of MFIs \autocite{berger2006capital,khachatryan2017performance}. On the other hand, commercialisation is also likely to affect social performance \autocite{d2017ngos}.

Moreover, research suggests that ownership structure influences the capital structure of firms. For example, institutional ownership is associated with higher leverage ratios \autocite{sun2016ownership}. Research further indicates that the choice of capital structure can also mitigate agency conflicts between shareholders and managers. High leverage or low equity value reduce the agency costs of outside equity \autocite{berger2006capital}.

Also, the nature of ownership- that is, institutional, or individual and the ownership structure may also influence the degree of monitoring and hence the financial and social performance. A high level of ownership dispersion is likely to be detrimental to monitoring, as opposed to large blocks of ownership by individual and institutional investors \autocite{becht1999blockholdings}. As noted earlier, however, such concentrated ownership may be harmful due to the tendency to extract private benefits of control.
Debt is a tool to reduce managerial discretion for two reasons. Primarily, debt reduces the free cash flows that managers can misuse or misapply. Besides, debt holders offer an additional layer of monitoring, and hence the view of banks and other financial intermediaries that provide debt as delegated monitors \autocite{diamond1996financial}. The presence of debt raises the threat of liquidation, where managers lose their salaries, perquisites, and even reputation \autocite{jensen1976theory}.

However, the entry of debtholders in transformed MFIs poses a new set of conflict: the agency conflicts between shareholders and debtholders. As providers of capital, both shareholders and debtholders have an interest in the performance of the firm. However, the shareholders have limited liability but remain residual claimants after settling the debt. On the other hand, the debt-holders earn a fixed income but could lose their investment if the firm fails. The concern for the debtholders is that the owners may use the borrowed money to make risky investments. The motivation for the risky investment is that if the investment succeeds, the owner keeps the residual, but if it fails, the debt-holder suffers substantial loses \autocite{chu2017shareholder}. As a result, the debtors also monitor the management intensely.

Given this analogy, the capital structure may moderate the balance between the achievement of social goals and financial performance, given that investors are assumed to prioritise self-interest. For debtholders who are recipients of a fixed return, the emphasis on the social mission of MFIs is likely to be a disadvantage. However, the effects of the commercialisation of microfinance are likely to be subject to several additional factors. These factors comprise but are not limited to the degree of share ownership vis-à-vis dispersion. Thus, the establishment of the influence of the changes in MFIs must take cognisance of a myriad of factors both within and without the organisation that may have a bearing on the different dimensions of performance by MFIs.

\hypertarget{capital-structure-theory}{%
\subsection{Capital Structure Theory}\label{capital-structure-theory}}

\noindent The capital structure theories attempt to explain the financing mix of a firm regarding the proportion of debt and equity in the firms' long-term funding. \textcite{modigliani1958cost} developed the most prominent theory of capital structure. According to this theory, the choice of the mix of debt and equity is irrelevant; that is, it does not affect the value of the firm. Later refinements of the theory to include taxation, and the costs of financial distress concluded that considering financial frictions; financing structure does matter. This section does acknowledge both the net operating income and the net income theories of capital structure. However, the section deals primarily with the \textcite{modigliani1958cost} capital structure theories which dominate the thinking around the capital structure. The section also provides a discussion of the pecking order theory.

The MM theory is the backbone of modern thinking on capital structure. Developed by Franco Modigliani and Merton Miller starting from 1958, the theory has four components, namely MM propositions I, II, III and IV. In MM 1, Modigliani and Miller viewed the value of a firm and its capital structure to be independent. Put another way, the cost of capital did not vary with changing the capital structure. Instead, the value of a firm was dependent on its expected performance and commercial risk \autocite{myers2001capital,chang2015reconsideration}. The proposition implied that an optimal capital structure did not exist.

MM II recognised the existence of corporate taxes. With corporate taxes, they concluded that the financing structure of a firm does matter because interest on the debt is an allowable deduction for taxation, while dividends on both common and preferred stock are not \autocite{dempsey2014m}. The variance in value between a levered firm and an all-equity firm is the interest tax shield, that is the tax-saving because of having paid interest on the debt. The inference is that the optimal capital structure is 100\% debt, which goes against the observed practice of corporate financing.

The MM III improved on MM II by integrating personal taxation. The inclusion of personal taxation changed the objective of the firm from that of minimising the corporate tax bill to that of minimising all taxes, both corporate and personal, which both the bondholders and shareholders face. Put another way the firm should arrange its financing to maximise the after-tax (both personal tax and corporate tax) income. Under this scenario, corporate borrowing is better if the after-tax income is higher than that without taxes, otherwise, it is advisable to borrow less. This approach introduced the concept of the relative tax advantage of debt. However, it is MM IV, also called the trade-off theory, that dominates the capital structure debate.

\hypertarget{trade-off-theory}{%
\subsubsection{Trade-off Theory}\label{trade-off-theory}}

\noindent MM IV was a refinement of the MM III, recognising that although debt has its merits, it raises the probability of financial distress. Thus, managers must strike a balance between the tax benefits of debt against the reality that a firm increases the likelihood of default, the higher the debt in the capital structure \autocite{de2011firms}. For this reason, MM IV is also called the trade-off theory. The costs of financial distress may be direct, for example, the legal costs of reorganisation. Indirect costs may include the loss of customers, suppliers, employees, receivables, and the loss associated with the fire sale of assets \autocite{ehrhardt2016corporate}. The following equation summarises MM IV.

\begin{equation}
V_{l} = V_{u} + T_{s} - FDS
\end{equation}

Where

\begin{itemize}
\tightlist
\item
  \(V_l\) is the value of a levered firm.
\item
  \(V_u\) is the value of an unlevered firm, that is, a firm with no debt in its capital structure.
\item
  \(T_s\) is the present value of the tax shield, the product of the interest expense and the corporation tax rate.
\item
  \(FDC\) is the present value of financial distress costs.
\end{itemize}

The introduction of financial distress has significant implications on the capital structure of firms. With financial distress costs, firms tend to avoid too much debt. Mature, profitable firms tend to rely on internally generated funds as do corporations with little cash flows and intangible assets. Firms with tangible assets for use as collateral have high leverage as do firms with high cash flows. Also, distress costs mean that firms would favour debt that is easily reorganizable, which favours bank loans rather than debt from public capital markets. For flexibility, firms tend to deal with few banks and opt for a few classes of debt \autocite{ehrhardt2016corporate}.

The trade-off theory has two important strands. As described above, the static trade-off theory seeks to balance between the tax advantage of debt and the potential costs of financial distress. However, subsequent research later demonstrated that under certain conditions, the tax disadvantage at the individual level offset the tax advantage of debt at the corporate level \autocite{bradley1984existence}. The research developments led to the introduction of an array of leverage-related costs such as bankruptcy costs themselves, agency costs of debt, loss of non-debt tax shields (e.g., accelerated depreciation and investment tax credits that reduce corporate tax liability), and the presence of personal taxes on dividends.

Thus, the calibration of the optimal capital structure must involve the balance between the tax benefits of debt on the one hand, and a range of leverage-related costs on the other. The choice of optimal capital structure then turned into an empirical issue investigating the significance of the various leverage-related costs. Much of this subsequent research indicates that the choice of an optimal capital structure is industry-specific, influenced by non-debt tax shields, and financial and trade cycles \autocite{miao2005optimal}. Further, the level of debt varies with industry output, plant closings, firms' entry into the industry, and even capital investments by firms \autocite{myers2001capital}.
The dynamic trade-off theory asserts that firms have a target capital structure depending on their specific characteristics. Thus, firms will issue securities that bring them closer to their target capital structures \autocite{elsas2014financing}. In other words, the theory asserts that mean reversion to the optimal capital structures characterises firms financing. Notable in this theory, though is that firms incur costs in adjusting their capital structures to their desired optimal levels. The speed of adjustment to the optimal structures is also subject to research, and a significant point of contention for the dynamic trade-off theory. There is, however, no evidence of a universal optimal or target capital structure in firms of industries \autocite{elsas2015dynamic}.

The starting point of the critique of the trade-off theory is the empirical soundness of the model. Whereas taxes are inevitable, bankruptcy is rare, and so it is not proper to give equal weights to the probabilities in the choice of capital structure. Under this scenario, firms should have much higher leverage than is observed in practice. Whereas the tradeoff theory explains why firms adjust their high leverage by buying back debt and not by issuing new equity, it is weak in explaining extended under-leverage by firms. The under-leverage could be cured by issuing debt and buying back shares, an adjustment process rarely observed in practice \autocite{myers1984capital}.
In explaining the leverage level of firms, the trade-off theory has a very low coefficient of determination and cannot satisfactorily explain differing leverage levels across apparently similar firms. Furthermore, the theory cannot explain the choice of leverage targets by firms. Lastly, \textcite{welch2004capital} notes that stock returns explain about 40\% of debt-equity ratio, a situation that should not occur if firms repurchased debt or issued equity to counteract the effects of stock returns of the debt-equity ratios. However, more recent empirical review of capital structure decisions points to a tendency for managers to issue stock when there has been a share price run-up, in support of the tradeoff theory and managerial market timing {[}elsas2014financing{]}.

\hypertarget{pecking-order-theory}{%
\subsubsection{Pecking Order Theory}\label{pecking-order-theory}}

\noindent \textcite{donaldson2000corporate}, originally published in 1961, laid the foundations for the pecking order theory, stating that ``Management strongly favoured internal equity as a source of new funds to the exclusion of external funds except for occasional unavoidable `bulges' in need for funds'' \autocite{myers1984capital}. \textcite{myers1984corporate} brought the pecking order theory to the fore. The theory contends that in raising capital, corporations are also subject to floatation costs- costs associated with the issuance of financial assets. The floatation costs compound the information asymmetry faced by investors. Most corporations, therefore, issue capital in a predetermined order usually starting with the cheapest. Most firms will start by raising capital internally using their retained earnings. When retained earnings are exhausted, firms resort to short-term marketable securities. Preferred stock, debt and common stock respectively are usually the least preferable \autocite{chirinko2000testing,fama2002testing}.

The assumption behind the pecking order theory is that managers have an information advantage over investors (information asymmetry). Under information asymmetry, managers will tend to issue risky securities when they are overpriced. Investors, aware of managers information advantage, interpret any issue of securities as a signal pointing the valuation of the company's securities. Investors thus discount the firms new and existing securities when new issues are announced \autocite{fama2002testing}. Usually, investors and by extension the capital markets react negatively to new stock issues \autocite{myers1984capital,chirinko2000testing}. The logic behind the adverse reaction is that if the prospects of the firm were favourable, there would be no need to dilute the ownership structure, and hence reduce the future earnings per share of the firm. On the other hand, capital markets usually react positively to the issue of debt.

A more complex modification to the pecking order theory assumes that firms are concerned with, not just current financing costs but also with future financing costs. Thus, to balance the present and future costs of financing, firms with substantial expected investments maintain leverage. The low leverage helps firms to avoid having to forego future lucrative investments and minimise any material costs of financial distress. It also allows firms to avoid financing such future lucrative investments with new risky securities.
The leverage allowance is called the reserve borrowing capacity. The reserve borrowing capacity allows corporations to have the flexibility to issue securities to take up attractive investment opportunities that arise by quickly issuing debt \autocite{ehrhardt2016corporate}. Furthermore, firms tend to set dividend policies that allow the firm to meet most of its immediate financing needs using internally generated funds. In this respect, the dividend ratio maintained by firms tends to be sticky; that is, it tends to be steady, and not exhibit sudden fluctuations unless the firm is in financial distress.

Although some of the pecking order theory's predictions are valid, there is no evidence that the theory is of first-order importance in determining the capital structure of firms. For example, there is a definite relationship between expected investment and book leverage. Profitability, on the other hand, is negatively related to leverage. However, \textcite{myers1984corporate} cites some firms that issue stock when they could instead issue debt. Thus, the pecking order theory does not explain everything in corporate financing.
Moreover, it is hard to define the reserve borrowing capacity as this will vary from one firm to the other. Finally, the pecking order theory depends on sticky dividends but gives no hint why dividends should be sticky.

\hypertarget{factors-influencing-the-choice-of-financing-sources-by-firms}{%
\subsubsection{Factors Influencing the Choice of Financing Sources by Firms}\label{factors-influencing-the-choice-of-financing-sources-by-firms}}

The capital structures adopted by firms tend to vary across countries of operation, asset structures and operational risks faced by firms. Given that firms operate in different macroeconomic and microeconomic environments, and have different growth options, competitiveness, legal, and tax frameworks, they must adapt their capital structures to remain competitive. Thus capital structures vary across industries, countries and firm sizes. Table 3.1 and Table 3.2 summarises the firm-specific and country-specific determinants of capital structure respectively. Other than \textcite{moyo2013modelling} which was the primary source of information, the table lists other pertinent literature sources.

\newpage
\begin{landscape}

\begingroup\fontsize{8}{10}\selectfont

\begin{longtabu} to \linewidth {>{\raggedright}X>{\raggedright}X}
\caption{\label{tab:unnamed-chunk-1}Firm-Level Capital Structure Determinants}\\
\toprule
FIRM.SPECIFIC.FACTOR & EXPLANATION\\
\midrule
\endfirsthead
\caption[]{\label{tab:unnamed-chunk-1}Firm-Level Capital Structure Determinants \textit{(continued)}}\\
\toprule
FIRM.SPECIFIC.FACTOR & EXPLANATION\\
\midrule
\endhead

\endfoot
\bottomrule
\multicolumn{2}{l}{\rule{0pt}{1em}\textit{Source: }}\\
\multicolumn{2}{l}{\rule{0pt}{1em}Authors synthesis of literature from Moyo (2013), Gwatidzo \& Ojah, 2009, and Ojah \& Ombati (2016) among other sources}\\
\endlastfoot
Asset tangibility & Firms can use tangible assets as collateral for debt funding. Thus, firms with more tangible assets usually have more debt at a lower cost than firms with less tangible assets (Campello and Giambona , 2011;
Ojah and Ombati , 2016).\\
 & \\
Profitability, dividend and equity repurchases, and earnings volatility & Profitable firms are usually large and mature with free cash flows to support debt repayment and thus hold more debt. Firms take up debt to minimise cash that managers could misallocate. The debtholders act as delegated monitors and further act to monitor management. Similarly levered are small fast-growing firms which generate non-debt tax shields from their capital expenditures. Thus, small but fast-growing firms have higher leverage.
Firms with higher volatility have a higher probability of financial distress than those with lower volatility. Thus, firms with high earnings volatility take up less debt. (Barclay and Smith , 2005; Gwatidzo and Ojah , 2009; Ojah and Ombati, 2016).\\
 & \\
Competitiveness & Unlike firms operating in monopolistic or oligopolistic market conditions, firms in competitive industries have lower debt. The high debt would make a firm lose competitive advantage in the product market given that interest increases operating leverage (De Jong et al., 2011)\\
\addlinespace
 & \\
Size and growth rate & A large firm that has high profits, more tangible assets, and limited growth options and thus, uses debt to reduce free cash flows to curb agency conflicts. Small firms, unlike large firms, need to maintain flexibility to avoid agency costs of underinvestments. Hence, they rely more on equity (Barclay and Smith, 2005; Barclay et al., 2006; Gwatidzo and Ojah, 2009).\\
 & \\
Bankruptcy, financial distress costs and credit ratings & Firms with more debt have a higher probability of financial distress and hence borrow less. The opposite is the case for firms with less debt. Firms with higher credit ratings have higher debt, unlike firms with low credit ratings that have low debt (Barclay and Smith, 2005; De Jong et al., 2011.\\
 & \\
\addlinespace
Firm access to capital markets & The better chance the firm has access to the capital markets, the easier it can raise both debt and equity. A firm’s access to capital markets is a function of its size and financial performance . High-quality firms tend to have more leverage as they have the option to raise both debt and equity with ease. Medium quality firms are usually bank-oriented and have medium levels of debt. Low-quality firms rely on private debt which is costly or unavailable. Thus, they tend to have low leverage (Barclay and Smith, 2005; Ehrhardt and Brigham, 2016; Ojah and Ombati, 2016).\\
 & \\
Internal funds deficiency and investment programs & A firm with internal funds deficiency that has good investment options will issue more debt instruments. On the other hand, a firm with adequate internal cash and poor investment prospects tends to issue equity (De Jong et al., 2011).\\
 & \\
State of stock and bond markets, and the degree of firm capital market orientation & If the stock market is bullish, the firm will issue equity as the firm maximises net cash issue proceeds. In a bear market, managers will favour debt.  Additionally, firms will issue debt when it is cheaper and less risky than equity, and thus the cost of debt (interest rate) is a significant determinant of capital structure (Ehrhardt and Brigham, 2016).\\
\addlinespace
 & \\
Tax factors & Debt gives a tax advantage to firms, but the costs of financial distress can outweigh the tax advantage of debt. The cut-off where the tax advantage of debt outweighs the costs of financial distress is not clear-cut. However, firms will always seek to maximise the interest tax shield while guarding against the costs of financial distress. Most firms will have debt in their capital structure to take advantage of the interest tax shield (De Jong et al., 2011).\\
 & \\
Peer or industry capital structure & Financing patterns of firms tend to follow the financing structures of their peers, especially the better-performing ones. Thus, firms in an industry tend to have similar financing structures (Ehrhardt and Brigham, 2016).\\
 & \\
\addlinespace
Managerial factors & Managers strive towards earning extra material rents (perquisites). Hence, they favour higher profitability and larger firms. Managers adopt capita structures that maximise their wealth. Research indicates that inside/ managerial ownership is positively related to higher leverage. However, other researchers find the opposite and the debate is still unsettled (Agrawal and Mandelker, 1987; Kim and Sorensen, 1986; Jensen et al., 1992).\\*
\end{longtabu}
\endgroup{}

\newpage

\begin{table}

\caption{\label{tab:unnamed-chunk-2}Country-Level Capital Structure Determinants}
\centering
\fontsize{8}{10}\selectfont
\begin{tabu} to \linewidth {>{\raggedright}X>{\raggedright}X}
\toprule
FACTOR & EXPLANATION\\
\midrule
Country macroeconomic conditions and business cycles & Business cycles affect the returns on equity. The market timing theory suggests that firms will issue equity when prices are high, that is during boom periods. Conversely, firms are more likely to issue debt in bear markets Al-Zoubi et al. , 2017).\\
 & \\
The structure, performance, and regulation of the capital markets. & Countries are either stock market-oriented or bank oriented. The level of stock market development, investment banking relations, corporate governance practices and the nature of investor protection laws, determine the degree of orientation. Most firms in bank oriented countries source their capital from debt and have high leverage. Firms in stock market-oriented firms tend to issue more equity and are less leveraged (Matias and Serrasqueiro, 2017; Ojah and Ombati, 2016).\\
\bottomrule
\multicolumn{2}{l}{\rule{0pt}{1em}\textit{Source: }}\\
\multicolumn{2}{l}{\rule{0pt}{1em}Authors synthesis of literature from Moyo (2013), Gwatidzo \& Ojah, 2009, and Ojah \& Ombati (2016) among other sources}\\
\end{tabu}
\end{table}

\end{landscape}

When evaluating the factors listed above in considering the transformation of MFIs, several implications surface. First, given that access to capital markets is a crucial determinant of a firm's capital structure, the level of capital market development could play a role in the decision of MFIs to transform. Similarly, the MFI size and growth rates could also be significant predictors of the decision by MFIs to transform. The factors listed above also explain the intertemporal variations in capital structure and the speed with which firms' capital structures adjust from temporary states of disequilibria.

\hypertarget{quantifying-the-performance-of-mfis}{%
\section{Quantifying the Performance of MFIs}\label{quantifying-the-performance-of-mfis}}

\noindent Given the double bottom line characteristic of MFIs, measuring their performance is complex and contested. Extant literature measures both financial and social performance separately. Even then, researchers cannot agree on a single scale. For instance, in measuring the financial performance of MFIs and indeed of any firm, there are a variety of metrics, ranging from the return on assets, return on equity, to profit margin ratios. Similarly, for social performance, metrics range from the number of women borrowers, the average loan size to loans granted to rural dwellers.

Notable, though, is that there is no single metric used to measure both the financial and social performance of MFIs mainly because the two measures lack a standard or common unit. This section examines the variety of measures used to measure both financial and social performance of MFIs and ends by describing the recent developments in the measurement of MFI performance and offers some concluding remarks and reflections.

\hypertarget{financial-performance-of-mfis}{%
\subsection{Financial Performance of MFIs}\label{financial-performance-of-mfis}}

\noindent Firms with sound financial performance metrics can meet their obligations as they become due \autocite{ehrhardt2016corporate}. For instance, financially healthy firms can pay their employees and suppliers, and leave a surplus that they distribute to the providers of capital and/or retain some cash to expand their operations. In the case of MFIs, the better financial performance also means that they can provide MF services without excessive reliance on donations and grants. Furthermore, MFIs may apply the profits generated to expand outreach to the poor at a lower cost, a move that researchers term as cross-subsidisation and mission expansion \autocite{mersland2010microfinance}. There are among several measures for quantifying the performance of MFIs, the most common are the financial ratios.
A commonly used financial ratio is the return on assets (ROA). ROA is the ratio of the MFIs net operating income and the assets.

\begin{equation}
ROA = \frac{Net Operating Income}{Assets}
\end{equation}

ROA is a simple ratio to calculate and indicates management's efficiency in utilising MFIs assets to generate returns for the providers of capital. The aim is to maximise the ratio and, at a minimum, generate enough revenue to cover the risk-free rate and compensate the providers of capital for the systematic risk facing the MFI \autocite{mersland2014microfinance}. ROA lacks a clear cut off point. Researchers often use operational self-sufficiency (OSS) to measure the financial performance of MFIs. OSS is an indicator of the ability of a firm to cover its expenses. There are two variants of OSS: OSS 1 and OSS 3. OSS 1 is defined as follows,

\begin{equation}
OSS1 = \frac{Operating Revenue}{Expenses on Funding, Loan Loss Provisions, and Operations}
\end{equation}

If OSS\textgreater1, then the MFI can cover its expenses and have some leftover for growth. At OSS=1, then the MFI is just able to meet its expenses with no leftover. At the opposite end, an \(OSS < 1\) means that the MFI cannot meet its expenses (Mersland and Strøm, 2014).
OSS 2 on the other hand, is defined as follows,

\begin{equation}
OSS1 = \frac{Operating Revenue}{Operating Expense}
\end{equation}

The calculation of OSS 2 does not factor in the cost of funding. According to \textcite{mersland2014microfinance}, eliminating funding costs is paramount given that funding sources vary a lot between MFIs and across regions. It means, therefore that managers cannot directly influence funding costs. Moreover, in the modern finance industry, operating costs are getting increasingly meaningless especially with the reduced role of physical infrastructure. The weakness of both OSS 1 and OSS 2 is that they ignore the cost of funding associated with equity \autocite{mersland2014microfinance}. The omission is understandable given the difficulty in determining the cost in developing countries with shallow financial markets. Financial self-sufficiency (FSS) is another commonly used measure. FSS goes beyond the OSS 1 by adjusting the accounting values to reflect the market values in line with fair value accounting rules. With the adjustment, the FSS then becomes,

\begin{equation}
FSS = \frac{Adjusted Operated Revenue}{Adjusted Expenses on Funding, Loan Loss Provisions, and Operations}
\end{equation}

\textcite{christen2001commercialization} developed the FSS. The researchers highlighted the need to adjust the accounting values for inflation and implicit and explicit subsidies \autocite{mersland2014microfinance}. Overall, financial ratios are transparent and standardised. However, some of the figures used are generated subjectively via management judgment, for instance, provisions for loan losses \autocite{ehrhardt2016corporate}. While financial sustainability measures inform about the financial status of the MFI, they do not indicate the extent to which the MFI is reaching the poor. The next section describes the measures of the second dimension of MFI performance: outreach.

\hypertarget{social-performance-outreach-of-mfis}{%
\subsection{Social Performance (outreach) of MFIs}\label{social-performance-outreach-of-mfis}}

The other primary goal of MFIs is to supply financial products to the poor, especially women and rural dwellers, in developing countries. The extent to which MFIs can reach the poor is called outreach and is the degree of social performance. The social performance of MFIs has two dimensions: depth and breadth. The depth of outreach refers to the extent of poverty of the clients served by MFIs. The breath, on the other hand, refers to the absolute number of potential MFIs clients reached, or the scale of operations \autocite{bibi2018new}.

Recognising the multi-dimensionality of MFIs, the social performance task force (SPTF) has developed several measures of social performance. Similarly, The French Comité d'Echanges, de Réflexion et d'Information sur les Systèmes d'Epargne-crédit (CERISE) also developed a set of social performance indicators (SPI)\autocite{mersland2014microfinance}. The SPTF outreach composite index takes into account the ``provision of financial services to the poor and the financially excluded, improvement of quality and appropriateness of financial services, and improvement of the economic and social conditions of clients. The index also considers that MFIs must strive to ensure social responsibility to clients, employees, and the community served''\autocite{mersland2009performance}. Similarly, the CERISE SPI composite index makes use of ``targeting and outreach, products and services, benefits to clients, and social responsibility.'' However, given the difficulty in understanding composite indices and controversies in weights assignment when constructing indices, researchers still use the individual measures of MFI outreach \autocite{chattopadhyay2017applicability}. Researchers use several significant proxies for breadth and depth of outreach as shown in the Table 3.3.

\begin{table}

\caption{\label{tab:unnamed-chunk-3}Individual Measures of the Social performance of MFIs}
\centering
\fontsize{8}{10}\selectfont
\begin{tabular}[t]{ll}
\toprule
METRICS & PROXY\\
\midrule
Breadth of Outreach & Loan portfolio; 
Credit clients; 
Average loan; Savings; savers.\\
 & \\
Depth of Outreach & Average loan; 
Average loan/ GDP per capita; 
Female borrowers; 
Rural borrowers.\\
\bottomrule
\multicolumn{2}{l}{\rule{0pt}{1em}\textit{Source: }}\\
\multicolumn{2}{l}{\rule{0pt}{1em}Mersland and Strøm (2014)}\\
\multicolumn{2}{l}{\rule{0pt}{1em}\textit{Note: }}\\
\multicolumn{2}{l}{\rule{0pt}{1em}\textsuperscript{*} The higher the value of breadth measures, the higher the breadth of outreach.}\\
\multicolumn{2}{l}{\rule{0pt}{1em}\textsuperscript{\dag} Also, the growth of the breadth values is a vital indicator of the direction of the industry.}\\
\multicolumn{2}{l}{\rule{0pt}{1em}\textsuperscript{\ddag} For average loan and average loan divided by GDP per capita, the lower the value, the higher the depth of outreach.}\\
\multicolumn{2}{l}{\rule{0pt}{1em}\textsuperscript{\S} For female and rural borrowers, the higher the number, the higher the depth of outreach}\\
\multicolumn{2}{l}{\rule{0pt}{1em}\textsuperscript{\P} Growth rates are also significant in this respect. See Mersland and Strøm (2010) for details}\\
\end{tabular}
\end{table}

The following commentary is on a few of the metrics in Table 3.3 above. Usually, women exhibit high levels of poverty and, thus, there is an overlap between indigent clients and the number of women. Furthermore, researchers argue that a loan made to a woman may benefit the household more and, hence, the focus on women. Also, given that researchers have found high repayment rates among women, it is easier for a woman client to acquire a loan. However, as men tend to start more informal businesses, it could be that men acquire loans from MFIs through women (mostly wives). Also, credit services to women may reduce the obligation of men to contribute to their families \autocite{mersland2014microfinance}.

There are two reasons why MFIs should focus on rural areas. First, income levels are lower in rural areas than in urban areas. With increased rural-urban migration, the labour force available for agriculture is reducing. Technology often fills the void left by reduced agricultural labour which calls for more investments. These technological improvements create demand for credit and other financial services \autocite{mersland2014microfinance}.

Researchers have used the average loan as a proxy for depth of outreach. The lower the average loan size, the higher the implied depth of outreach \autocite{d2017ngos}. However, the metric has several shortcomings. First, the average loan size may increase because the financial position of the clients has increased. Also, accumulation of loan arrears may also lead to higher loan size. Moreover, even where an MFI places a cap on the loan size a client can take, the clients may respond by taking loans from multiple institutions as experience from Nicaragua and India shows \autocite{bastiaensen2013after}. Further, the average loan size is affected by extreme values of some loans and also fails to consider the cross-subsidisation benefits where MFIs give massive loans to some individuals and institutions to expand outreach to the poor \autocite{chattopadhyay2017applicability,bibi2018new}. Similar shortcomings also plague the average loan size to GDP per capita metric. To address the weaknesses of both the traditional measures of the social performance of MFIs,\textcite{bibi2018new} develop a social performance metric based on market shares held by MFIs. They define their measure of the breadth of outreach as follows.

\begin{equation}
MSB_{ij} = \frac{NAB_{ij}}{TAB_j}
\end{equation}

Where \(MSB_{ij}\) represents the measure for the breadth of outreach (the market share of microfinance borrowers).
- \(NAB_ij\) refers to the number of active borrowers in MFI i in country j.
- \(TAB_j\) refers to the total number of MFI borrowers in country j.
They further define the depth of outreach as,

\begin{equation}
MSBA_{ij} = \frac{\frac{NAB_{ij}}{TAB_{j}}}{\frac{A_{ij}}{TA_j}}
\end{equation}

\begin{itemize}
\item
  Where \(MSBA_{ij}\) measures the depth of outreach. It is defined as the market share of the number of borrowers, computed as the market share of borrowers scaled by the market share of assets.
\item
  \(NAB_{ij}\) is the number of active borrowers in MFI i in country j.
\item
  \(A_{ij}\) is the total assets of MFI i in country j.
\item
  \(TA_j\) refers to the total MFI assets in country j.
\end{itemize}

Extreme values of loans influence a social performance measure such as the average loan size. The new measure of depth developed by 2bibi2018new corrects for the deficiency by using relative loan sizes in the place of absolute loan sizes. Furthermore, the value of the metrics falls between zero (0) and one (1), that strengthens comparison across MFIs. Also, the empirical tests of the suggested measures show that they provide a better measure of social performance than the traditional metrics \autocite{bibi2018new}.

However, the metrics are country-specific and are not useful in the cross country, regional and global comparative contexts. Furthermore, the measures utilise a single dimension of social performance and ignore financial performance. It means that the new metrics do not present a comprehensive view of the performance of MFIs. More importantly, most researchers examine the financial performance, and the social performance of MFIs separately. A measure of the performance of MFIs that incorporates both the financial and the social metrics would be advantageous. The next section examines the recent developments in the measurement of the performance of MFIs.

\hypertarget{developments-in-assessing-performance-of-mfis}{%
\subsection{Developments in Assessing Performance of MFIs}\label{developments-in-assessing-performance-of-mfis}}

Given the shortcomings of the traditional measures of the performance of MFIs, some researchers have suggested alternate measures. \textcite{cervello2019social} suggest a multidimensional measure of performance that simultaneously captures both the financial and social performance of MFIs. The measure that they have developed captures the trade-off between financial performance and social performance that has raised concerns about possible mission drift by transformed MFIs.

The measure suggested utilises the goal programming concept developed by \autocite{charnes1978measuring}. Researchers interpret goal programming under the philosophy of satisfaction with multiple objectives, unlike the mathematical optimisation techniques that have a single objective. The central idea is that individuals desire to minimise the non-optimization of (multiple) goals. The methodology, however, recognises that it is hard to achieve all goals simultaneously.

Extending the goal programming beyond the model of Charnes, et al., the solution suggested by \textcite{cervello2019social} combines different goal programming models to pick combinations of performance that have maximum consensus, and penalising combinations that have a high degree of conflict. Alternatively, the researchers could pick the combinations that have a maximum conflict against those that follow the general trend without loss of meaning. The score is the sum of all weights, which has a maximum value of one, provides the degree of the non-achievement of the set of goals, in this case, financial performance and social performance.

Another notable study by \textcite{chattopadhyay2017applicability} makes a case for the adoption of a two by two (2x2) classification matrix that captures both financial and social performance. The matrix gives a code of one (1) to the MFI that meets both the financial and social objectives. The matrix codes other cases as zero (0). Table 3.4 below illustrates the classification. The researchers point out that by applying the logistic regression (LR), a variant of the generalised linear regression model, based on the matrix could shed light on the factors that affect the ability of an MFI to achieve (or fail to achieve) the twin objective of financial performance and outreach. The advantage of the proposed metric is its ability to capture both the financial and social performance in one score.

\begin{table}

\caption{\label{tab:unnamed-chunk-4}Classification Matrix: Joint Financial and Social Performance of MFIs}
\centering
\begin{tabular}[t]{llll}
\toprule
  &   & Achieves\_Financial\_Goals? & Achieves\_Financial\_Goals?\\
\midrule
 &  & YES & NO\\
Achieves\_Social\_Goals? & YES & Class 4 (SS) & Class 2 (FS)\\
Achieves\_Social\_Goals? & NO & Class 3 & Class 1 (FF)\\
\bottomrule
\multicolumn{4}{l}{\rule{0pt}{1em}\textit{Source: }}\\
\multicolumn{4}{l}{\rule{0pt}{1em}Adapted from Chattopadhyay, Manojit, and Subrata Kumar Mitra (2017)}\\
\multicolumn{4}{l}{\rule{0pt}{1em}\textit{Note}}\\
\multicolumn{4}{l}{\rule{0pt}{1em}\textsuperscript{1} In labelling the classes, we start with financial sustainability followed by social performance.}\\
\multicolumn{4}{l}{\rule{0pt}{1em}\textsuperscript{2} The letters F and S stand for Fails (F) and Succeeds (S), respectively.}\\
\multicolumn{4}{l}{\rule{0pt}{1em}\textsuperscript{3} For instance, FS means the MFI fails (F) financially but succeeds (S) socially.}\\
\end{tabular}
\end{table}

\hypertarget{mfis-transformation-and-performance-empirical-review}{%
\section{MFIs Transformation and Performance: Empirical Review}\label{mfis-transformation-and-performance-empirical-review}}

\noindent With time, a substantial number of MFIs converted to commercial entities, and empirical analysis on the effects of the transformation was then possible \autocite{abeysekera2014sustainability,mia2017mission,d2017ngos}. Some researchers hold that mission drift is bound to occur in transformed MFIs as better-off clients crowd out the poor in the access to MFI credit \autocite{hishigsuren2006transformation}. Other scholars and practitioners hold the opposite view. This sections examines the arguments on either side and additional factors that may exarcerbate or reduce the extent of mission drift.

\textcite{campion1999institutional} argue that the presence or absence of mission drift in a transformed MFI is partly a corporate governance issue, and an outcome of the challenges of the scaling up of MF services. They note that good corporate governance allows the management to balance between financial performance and outreach. An alternate view is that mission drift is a misinterpretation of cross-subsidization and progressive lending. Thus, MFIs may reach out to clients who are better off to cross-subsidize loans for the poor and financially excluded \autocite{abeysekera2014sustainability}. Some research finds managers were not aware of the presence of mission drift as is quantified using the standard metrics \autocite{hishigsuren2007evaluating}. The finding underlines the view by \autocite{marti2016financial} that different social groups such as employees, management, and MFI clients are likely to have different views, including varying definitions of social welfare. Thus, the presence or absence of mission drift may not arise out of deliberate management decisions, but instead, out of conflicting viewpoints on social welfare between researchers, policymakers, and practitioners.

Part of the problem associated with the research on mission drift is the variety of proxies used to measure social performance. \textcite{mersland2010microfinance} and many other researchers quantified social performance by the average loan balance per borrower and the ratio of average loan balance per borrower to GNI. Some researchers have reservations regarding the use of the average loan size and the average loan size scaled by GNI per capita as proxies for mission drift. The critics argue that higher average loan size could be an indicator of progressive lending and cross-subsidization. Also, a dynamic economy may benefit SMEs, which, in turn, may demand more significant loans. They also point out that regions with relatively low proportions of indigent clients might be wrongly perceived to be drifting from their mission. Also, as noted, a few large organizations and individuals have large loan balances with MFIs that may bias the average loan balance upwards without necessarily indicating the presence of mission drift. \autocite{armendariz2011mission} further argue that the observed higher average loan sizes by MFIs could arise from the ``interplay between their mission, the cost differentials between poor and unbanked wealthier clients, and region-specific clientele parameters'' (pp.~341).

Consequently, researchers have suggested alternate measures of mission drift including, among others, the number of women borrowers, number of loans, X-efficiency scores, yields on the gross loan portfolio, and the number of borrowers in rural areas. However, most of the studies highlighted below use the average loan size, usually scaled with the GNI per capita, as a measure of mission drift. Table () captures additional proxies for depth and breadth of outreach, their strengths and shortcomings. For lack of better metrics, researchers continue to use them.

\hypertarget{empirical-studies-demonstrating-mission-drift}{%
\subsection{Empirical Studies Demonstrating Mission Drift}\label{empirical-studies-demonstrating-mission-drift}}

\noindent Some researchers hold that financial performance and social performance by MFIs are negatively related. This view is central to the mission drift proponents who assert that the pursuit of financial goals is incompatible with the achievement of social objectives. Early empirical research in support of the hypothesis includes \textcite{christen2001commercialization} who explored whether transformation made MFIs abandon indigent clients. They noted that transformed MFIs often had higher average loan sizes and higher average loan size scaled by GNP per capita compared to the non-transformed MFIs. Supporting this position, \textcite{kent2013bankers} argue that commercial banking logic has, over time, displaced the social roles of MFIs targeted at addressing financial exclusion and poverty.

Recently, \textcite{mia2017mission} also arrived at a similar conclusion by applying both static and dynamic panel data estimation techniques on a sample of 169 Bangladeshi MFIs over the period 2009-2014. The researchers conclude that mission drift may occur when MFIs seek higher financial returns. However, they note that cost efficiency neutralizes the extent of mission drift. Thus, the degree of cost efficiency is an essential moderating variable for MFIs that experience mission drift. Hence, if transformation results in an improvement in cost efficiency, then mission drift should not be a cause for concern.

Also, \textcite{cull2011does} examine the existence of mission drift in MFIs using instrumental variables regression and treatment effects to control for non-random assignment of supervision. They find that profit-oriented MFIs respond to supervision by reducing outreach to the poor and other customers that are costly to reach and instead focus on sustainability. On the contrary, MFIs running the NGO model that has a weaker commercial focus tend to maintain outreach at the expense of profitability. The researchers also find that MFIs curtail outreach to women and customers who are costly to reach. They further find that regulated MFIs with a weaker commercial focus maintain outreach at the expense of profitability. The research supports \textcite{cobb2016funding} who find that private providers of funding emphasize profitability, whereas the public, donors and governments stress social performance.

\begin{enumerate}
\def\labelenumi{\arabic{enumi})}
\tightlist
\item
  goes further and explores the extent of mission drift by benchmarking the X-Efficiency scores based on revenues, depth of outreach (average loan size), and breadth of outreach (number of loans). The researchers found that MFIs that have a high depth of outreach also have the highest levels of breadth of outreach and profits for the same input mix. The study further uncovers significant trade-offs between financial sustainability and social performance. Similarly, \textcite{mersland2010microfinance}, using panel data estimations with instruments, find that high-cost MFIs, fund more individual borrowers, fewer women and focus more on urban areas. Low-cost MFIs display the reverse. These results indicate that the depth of outreach and financial sustainability could be pursued together if MFIs emphasise improving their efficiency by managing their costs.
\end{enumerate}

Lastly, \textcite{dorfleitner2017microfinance} examined the factors that cause MFIs to fail socially. First, they find that MFIs with better portfolio quality are less prone to social failure. Second, MFIs with a higher percentage of donations and regulated MFIs also have a lower possibility of social failure. Finally, they find that fast-growing MFIs are more likely to fail socially. The results indicate that financial sustainability, as proxied by portfolio quality, could play a role in the achievement of the social mission given that commercial MFIs have better portfolio quality \autocite{tchakoute2010there}, in support of the mission expansion and cross-subsidization school of thought. Similarly, expansion of MFIs, usually resulting in a breadth of outreach, may be detrimental to the outreach to the poorest.

\hypertarget{evidence-against-the-existence-of-mission-drift}{%
\subsection{Evidence Against the Existence of Mission Drift}\label{evidence-against-the-existence-of-mission-drift}}

\noindent On the other hand, \textcite{abeysekera2014sustainability} find no significant evidence of mission drift in their study, upon using static panel data model based on a sample of MFIs in Vietnam. For robustness, they measure outreach using a variety of measures: average loan size, interest cost to clients, and breadth of outreach. However, they uncover the presence of mission drift when they apply dynamic panel data models. The researchers conclude that the initial conditions (such as the initial average loan size, and micro-regional differences) and other time-varying factors play a role in the degree of mission drift experienced by MFIs.

Kar (2013) further invalidates the concerns about mission drift by using a sample of 409 MFIs from 71 countries. He uses a variety of proxies as well to measure mission drift, including average loan size, number of women borrowers, and the depth of outreach-profitability linkage. Using fixed effects model, and controlling for size, age, loan delivery method, legal status and geographical location, the study fails to confirm the presence of mission drift in the sampled MFIs. Thus, although mission drift could be MFI-specific, the researchers point to the lack of dynamism in age and the failure to decompose the size of MFIs into subsidised and unsubsidized components could be the significant drawbacks in establishing the presence of mission drift.

Finally, \textcite{roberts2013profit} also find no evidence of mission drift. In the study involving 358 MFIs and using OLS model, he finds that although commercial MFIs reduce service to rural clients, they serve more women and maintain the same average loan size. The results imply that sustainability enables MFIs to reach more clients without necessarily locking out the poorest, a position also held by \textcite{nurmakhanova2015trade} in their study based on a global dataset of MFIs.

\textcite{abdulai2017trade} uncover mixed results on mission drift. They find an insignificant link between sustainability and outreach by applying correlation analysis and the fixed-effects model. They uncover an insignificant relationship between sustainability (as measured using the operational self-sufficiency, OSS) and the average loan size scaled by the per capita GNI. However, they find the depth of outreach, measured using the number of borrowers to be positively related to sustainability. These results confirm the concerns by researchers regarding the appropriate measures of mission drift.

\hypertarget{culture-and-mission-drift}{%
\subsection{Culture and Mission Drift}\label{culture-and-mission-drift}}

\noindent \textcite{manos2014determinants} extend the research by \textcite{abdulai2017trade}, exploring the place of culture in mission drift using a sample of 800 MFIs in 30 countries for the period 2000-2010. Using two-step fixed effects panel regression2, they find that MFIs that operate in cultures characterized by avoidance of high uncertainty and more future orientation perform better socially but score poorly on financial metrics. The results are in contrast with those of cultures with assertiveness and performance orientation that score better financially. Also, MFIs operating in cultures with a high score on gender equity focus less on extending financial services to the poor, as opposed to those operating in cultures with high power distance levels. Thus, cultural factors specific to countries and regions could also explain the regional variations in the degree of mission drift by MFIs.

\hypertarget{mission-drift-and-globalisation}{%
\subsection{Mission drift and Globalisation}\label{mission-drift-and-globalisation}}

\noindent Moreover, in evaluating the role of globalization on the performance and mission drift of MFIs, \textcite{forkusam2014does} find that the ratio of FDI to GDP is positively related to the average loan size, and inversely with the number of borrowers. The study applies the fixed effects panel regression to control for omitted time-invariant features such as country and culture. These relationships uncovered in the study suggest that increased globalization and the ensuing rise in international capital flows may result in MFI mission drift. The implication is that international capital in search of financial returns has little place for social performance by MFIs. It means, therefore, that international commercial capital flowing to MFIs is frought with mission drift.

\hypertarget{the-place-of-individual-credit-officers-in-mission-drift}{%
\subsection{The Place of Individual Credit Officers in Mission Drift}\label{the-place-of-individual-credit-officers-in-mission-drift}}

\noindent Interestingly, \textcite{beisland2019commercialization} take the argument about mission drift further by arguing that the behaviour of credit officers could also explain mission drift. Their study used a mixed-methods approach. First, the researchers ran correlations and tested for differences in the mean number of clients using t-statistics and chi-square statistics. Next, they ran a pooled OLS regression and random effects regression with standard errors. To complement the quantitative methods, the researchers obtained qualitative data from focus group discussions with credit officers and managers of the MFIs.

They found a positive relationship between credit officers experience in years and the size of the loans they grant. In other words, the more experienced a credit officer is, the larger the loan size they are likely to approve. Furthermore, the researchers uncovered that credit officers with more experience approve fewer loans to the young and the disabled, in line with \autocite{labie2015discrimination}. To start with, due to the information opaqueness of MFI clients, MFIs usually give powers to credit officers to accept or reject loan applications. The subjective preferences of credit officers could then be at play in their decision making.

The preferences of credit officers could be due to several reasons. First, most MFIs have introduced incentive-based pay for credit officers and managers. The negative relationship between the compensation and risk could make credit officers shun high-risk clients (like the poor) to reduce the risk in their loan book. The results of the study are robust even after controlling for gender, education, marital status and branch affiliation. Consequently, the research recommends the training of credit officers to check the ``personal'' mission drift uncovered by the study. The results of the study suggest the need for a careful examination of the MFI-specific factors that may drive mission drift, and which researcher often miss when they use pooled global datasets.

In a similar study conducted in China, \textcite{jia2016commercialization} examine the relationship between the personal characteristics of loan officers and the size and quality of their loans. The study applied fixed effects regression analysis at the loan officer level and assessed the period after the MFI had adopted the commercial model. The researchers found that loan officers with a career background in farming and local government were less prone to mission drift (that is, the average loan size of their portfolio did not rise), and did so without compromising portfolio quality. Thus, socially conscious MFIs could do with a human resource background check for potential recruits for loan officer positions in a bid to curb mission drift. In addition to staff training, background checks on employees, \textcite{ramus2017stakeholders} also argue that stakeholder engagement and social accounting are viable ways to address mission drift among social enterprises, which includes hybrid organizations like MFIs.

\hypertarget{transformation-and-the-efficiency-mfis}{%
\section{Transformation and the Efficiency MFIs}\label{transformation-and-the-efficiency-mfis}}

\noindent The other strand of literature has examined the relationship between efficiency and the social performance of transformed MFIs. Efficiency here refers to the ratio of output to inputs by MFIs. An efficient MFI should produce the maximum possible outputs given a set of inputs. Researchers have applied several techniques to measure efficiency. However, data envelopment analysis (DEA) and stochastic frontier analysis (SFA) are the dominant techniques.

DEA is a non-parametric, a-theoretical technique that uses sets of multiple inputs and multiple outputs to estimate the best production frontier. Initially proposed by \textcite{charnes1978measuring}, DEA involves the construction of a virtual output over a virtual input without defining a production function. There are two variants of the DEA. The original DEA-CCR (CCR for Charnes, Cooper, and Rhodes) assumes constant returns to scale. \textcite{banker1984some} developed DEA-BCC (BCC for Banker, Charnes, and Rhodes) which allows for variable returns to scale. The advantage of DEA is that it does not require the specification of the functional form for the calculation of efficiency scores \autocite{cook2014data}.

Furthermore, DEA is not data demanding and can handle small samples and multiple inputs and outputs without having to standardise them. DEA also avails explicit improvement targets for inefficient decision-making units (DMUs) \autocite{paradi2017data}. The downside is that DEA does not handle measurement errors well and cannot handle noise in the data, treating all deviations as inefficiencies. It also assumes that firms are homogeneous and use a similar set of inputs to produce outputs making comparisons of DEA scores more meaningful in a single country setting \autocite{lebovics2016financial}. Lastly, the DEA assumption that firms have control over inputs and outputs can be faulty as the environment, and the government can override the management in resource allocation.

Stochastic frontier analysis (SFA), on the other hand, is a parametric method. \textcite{aigner1977formulation}; \textcite{meeusen1977technical} simultaneously formulated SFA. Subsequently, numerous researchers have used SFA to evaluate production, cost, revenue, profit and other models of goal attainment. The original formulation of SFA is as follows.

\begin{equation}
\beta{x} + v - u
\end{equation}

In the equation, \(Y\) is the observed outcome, and \(\beta{x} + v\) represents the optimal or stochastic frontier, for example, the maximum production or the minimum cost. In the optimal frontier, \(\beta{x}\) is the deterministic part whereas \(v\) is the stochastic part. The amount by which an entity fails to reach the stochastic frontier is the level of inefficiency that \(u\) captures. There are many extensions of SFA that are beyond the original specification. SFA allows for technical inefficiency but also provides for random shocks beyond the control of management that may affect output \textcite{kumbhakar2015practitioner}. The limitation of SFA is the assumption that there exists a functional form. Furthermore, it is hard to specify the error structure, especially the probability distribution. Lastly, the premise of continuity inherent in SFA and other econometric approaches to measuring efficiency may lead to approximation errors \autocite{cullinane2006technical}.

Thus, for given MFI inputs, how much is the MFI able to generate regarding financial returns and outreach to the poor. \textcite{lebovics2016financial} explore whether there is a trade-off between financial efficiency and social efficiency of MFIs in Vietnam. They use DEA to quantify both financial and social efficiency. The study uses regression analysis and uncovers no trade-off between financial efficiency and social efficiency. The results do not favor either the poverty-lending view or the financial-systems view approaches to MF provision. These results are similar to those of \textcite{bedecarrats2012combining} who argue that MFIs could achieve both financial and social performance as long as the MFI has practical social performance management strategy. These views reinforce the argument by \textcite{abeysekera2014sustainability} that mission drift could be a corporate governance problem.

\textcite{haq2010efficiency} examined the efficiency of 39 MFIs from Africa, Asia and Latin America using the DEA approach. The objective was to uncover the MFI types (Bank MFIs or NGO MFIs) that were most efficient at simultaneously minimising costs (hence maximising financial performance) and reaching out to poor households. Under the production approach, they found that NGO MFIs are the most efficient at reaching the dual objectives. However, under the intermediation approach, the research indicates the reverse. The results show that Bank MFIs could be better suited to lower costs associated with defaults. However, NGO MFIs achieve the dual objectives of financial sustainability and performance better.

Research by \textcite{hermes2011outreach} used a sample 1,300 observations and measured efficiency using SFA. The study found a negative relationship between outreach and efficiency of MFIs. More specifically, the study found that those MFIs with low average loan balance- a measure of the depth of outreach, are less efficient. Moreover, the study finds a negative relationship between breadth of outreach as measured using the number of women borrowers and the efficiency of MFIs. The argument presented to explain the negative relationship between efficiency and outreach is that transformation of MFIs generates costs which drive down outreach, especially to high-cost clients \autocite{cull2011does}. \textcite{roberts2013endogeneity} also concludes that commercialization of MFIs moves them further away from their efficient frontier of social performance. Also, there is a direct relationship between commercialization, operational costs, and interest rates.
Similarly, \textcite{abate2014cost} also use the SFA approach and find that there is a trade-off between outreach and the efficiency of MFIs in Ethiopia. The researchers first run an SFA regression and obtain the error term, µ, which is the measure of inefficiency. Next, they regress the inefficiency score against average loan size, women borrowers, and organization form, while controlling for age and size of the MFI.

\textcite{serrano2014microfinance} also recommend the need for cost efficiency by transformed MFIs in the face of reduced donor support. Their research draws on the Pareto 80/20 principle and logistic regressions on data for both drifted and non-drifted MFIs. The researchers draw parallels with the dot-com industry and recommend that MFIs should focus on raising their efficiency levels instead of raising prices (interest rates). Finally, they recommend a business model based on high turnover, instead of wide profit margins to drive MFI sustainability. However, given the intense competition in the MF sector, it is not clear whether the high turnover, low margin MFI business model is feasible.

\hypertarget{capital-structure-and-the-performance-of-mfis}{%
\section{Capital Structure and the Performance of MFIs}\label{capital-structure-and-the-performance-of-mfis}}

\noindent Before transformation, the presence of donations in the capital structure of MFIs was indicative of an emphasis on the social mission. It is, however, unclear whether the entry of commercial capital could overwhelm the social orientation of MFIs. From the business strategy perspectives, it is not apparent why commercially driven individuals and entities should care about the social performance of the MFIs. However, some researchers posit that corporate social responsibility (CSR) activities serve to resolve information asymmetries between firms and creditors. Hence, firms with a social mission have higher leverage and exhibit a faster speed of adjustment to the target capital structure \autocite{yang2018does}.
The research output on the relationship between leverage and financial performance is mixed. Some researchers have found the relationship between leverage and financial performance to be positive. For instance,\textcite{detthamrong2017corporate} find such a positive relationship in their study of Thai firms using OLS regression. Also, \textcite{abor2007debt} using data for SMEs from Ghana and South Africa and applying panel regressions that debt level is directly related to financial performance as measured by Tobin Q.

The other studies that also conclude that a positive relationship between capital structure and financial performance exists include \autocite{abor2005effect,berger2006capital,gill2011effect}. For instance, \textcite{fosu2013capital}, using fixed effects model and GMM finds that leverage is positively related to financial performance, with competition-enhancing the relationship. Most of these studies fail short of explaining the reasons behind the relationship. The study by \textcite{chakravarty2015role} that finds that MFIs with private funding are better able to screen borrowers and monitor borrower repayment rates. Consequently, such MFIs have lower portfolio risk and fewer delinquent loans. The results of this study could explain the positive relationship between capital structure and financial performance of MFIs.

Other studies, mostly based on emerging markets point to a negative relationship between debt and financial performance. The results imply that the costs of financial distress outweigh the benefits from the tax shields, at least in developing countries \autocite{le2017capital}. Similarly, \textcite{zeitun2014capital} also uncovered a negative relationship between leverage and both the accounting and market measures of performance of firms in Jordan. Similarly, \textcite{hamid2015capital} uncover a negative relationship in their study based on family firms. These results could be indicative that for MFIs, the costs of potential financial distress exceed the benefits generated by the interest tax shields. The introduction of private capital could also make managers of MFIs risk averse and thus set aside much higher than optimal precautionary savings to deal with any potential financial distress, as do MFIs facing subsidy uncertainty \autocite{armendariz2013subsidy}.

Midway between the two extremes, some researchers posit that capital structure can have both positive and negative effects on a firms financial performance. Such researchers have proposed that at low levels, firms reap the benefits of the tax shield, which gets overwhelmed by the costs of financial distress beyond a limit of leverage. The study by \textcite{lin2011does} is an example of research supportive of this position. \textcite{margaritis2007capital} also arrived at similar results by applying the quadratic functional form. From the discussion, the research on the link between capital structure and the financial performance of firms is inconclusive even in the mainstream corporate finance. The mixed results point to critical moderating variables on the link between capital structure and financial performance. The variables could relate to firm-specific factors and cross-country heterogeneity.

Nevertheless, researchers have extended the study of the link between to capital structure to financial performance to MFIs. \textcite{khachatryan2017performance}, find a link between the level of deposits in the previous year and financial performance using data from Eastern Europe. The positive relationship indicates that deposits could lower the cost of capital and hence increase sustainability. Further, they find that concessional loans do not significantly affect financial performance. Social loans and subsidies, however, are associated with lower profitability. However, the study does not delve into the possible impacts of equity capital and debt on the financial performance of MFIs. Filling this gap, \textcite{ayele2015microfinance}, using data for MFIs operating in Ethiopia, Kenya and Uganda found a negative relationship between debt-to-equity ratio and financial viability. Although these studies differ in countries covered, institutions considered, methodologies and data representation, the lack of a consensus makes policy-making difficult at the institutional and national levels.
In their research in the context of social performance, \textcite{khachatryan2017performance} find a positive relationship between social grants and the depth and breadth of outreach using seemingly unrelated regressions. Further, they found that concessional loans are positively related to outreach, as do social loans and subsidies. Another study by \textcite{hoque2011commercialization} found that leverage has adverse effects on outreach, productivity, and risk. They note, for example, that although the absolute numbers of indigent clients reached by MFIs increase, the proportion goes down due to the increased cost of capital, which implies higher default rates and hence higher portfolio risk.

On the contrary, \textcite{kyereboah2007determinants} applied pooled OLS and the fixed effects model on a panel dataset of MFIs in Africa and found a positive relationship between leverage and financial performance. The researcher also found that leveraged MFIs enjoys economies of scale, allowing them to deal with information asymmetry and hence lower portfolio risk. Similarly, \textcite{dorfleitner2017access} examined the factors that influence access to debt capital by MFIs using probit and Poisson regressions. The study uncovered a positive relationship between MFI access to MF Investment Vehicles (MIV) funding to maturity (measured by age, size of MFI and debt-to-asset ratio) and performance. For instance, more mature MFIs easily access debt capital from MIVs, as do firms with better financial performance regarding portfolio quality. Furthermore, MFIs that retain focus on their social mission have better access to debt financing. These results indicate that mature MFIs with better financial and social performance should then have more debt in their capital structure than do firms that do not meet the criterion.

\hypertarget{subsidies-and-grants-and-the-performance-of-mfis}{%
\section{Subsidies and Grants and the Performance of MFIs}\label{subsidies-and-grants-and-the-performance-of-mfis}}

\noindent It appears that the presence of donor funds is also a significant driver of social mission achievement, either because it allows MFIs to lower their cost of capital, reduces the pressure on the management to make profits or that donors emphasize on the outreach to the poor. Overall, the researchers note that there is a thin line between mission drift and cross-subsidization, which makes it difficult for researchers to distinguish them empirically. However, when \textcite{d2017ngos} use the event study methodology, which addresses the concerns highlighted above, they still uncover a sharp rise in average loan size attributable to the transformation of MFIs.

In their analysis of the effects of aid volatility on mission drift, \textcite{d2017ngos} find that average loan size is inversely related to aid volatility. On the other hand, interest rates charged on loans are directly related to aid volatility. The researchers conclude that most MFIs have internalized and use average loan size as a signalling device to demonstrate a commitment to the social mission. They also use interest rates to adjust for uncertainty.@morduch2005economics argue that subsidy uncertainty results in mission drift among MFIs as they seek to build up precautionary savings to cushion themselves financially by serving wealthier clients.

For MFIs, subsidies form a substantial percentage of the ownership structure of MFIs. Except for 23\%, the other MFIs have a portion of subsidies and grants in their capital structure \autocite{d2017ngos}. Analysing the effects of capital structure on the performance of MFIs is more complicated in the presence of the subsidies and grants. Research indicates that donors direct most of their funding to those MFIs that meet their social mission. Even for the MFIs that have not transformed, there is a significant reduction in the magnitude of donor support. Consequently, these NGOs have adopted various mechanisms to run their operations in the face of the reduced financial support from donors.

\textcite{d2017ngos} note that in Africa and Asia, MFIs cope with the lack of donations by charging higher interest rates. In Latin America, on the other hand, MFIs with less donor funding tend to serve fewer women. Unsubsidized MFIs in Eastern Europe and central Asia serve less indigent individuals. Overall, the researchers find that the lack of subsidies worsens the social performance of MFIs, but the effects vary across regions. Also, they argue that there exist trade-offs even among the social measures, for instance between service to women versus interest rates charged \autocite{d2013unsubsidized}. However, researchers have not arrived at a consensus regarding the existence of mission drift for transformed MFIs.

Furthermore, \textcite{cozarenco2016type} analyze the characteristics of MFIs that supply savings products on a global dataset of 722 MFIs for the years 2005-2010. They find that MFIs that receive donations and subsidies have a lower tendency to offer savings products. They conclude that subsidies and grants generate negative externalities on product diversification by MFIs.

Some research also indicates that the source of the donations matters. For instance, \textcite{chakravarty2015role} find that MFIs with more private donor funds tend to have lower rates of the portfolio at risk and fewer delinquent loans. Thus, the management of such MFIs improves significantly with private funding, presumably due to the monitoring role that private funders play. In a related study, \textcite{hudon2010management} also notes that better-managed MFIs tend to attract more donations. Overall, the presence of subsidies appears to have a significant link with the social performance of MFIs. However, the presence of subsidies may discourage the development of products, such as savings. Lastly, the source of the funding matters, with MFIs relying on public subsidies faring worse on both metrics.

\hypertarget{what-are-the-determinants-the-capital-structures-of-transformed-mfis}{%
\section{What are the Determinants the Capital Structures of Transformed MFIs?}\label{what-are-the-determinants-the-capital-structures-of-transformed-mfis}}

Research on the determinants of the capital structure of MFIs is scanty. Most of the available literature deals dwell on the capital structures of mainstream corporations. For financial institutions, which come close to MFIs, \textcite{gropp2010determinants} found that the determinants of capital structure for financial firms are like those of non-financial firms and that capital regulations and buffers are of secondary importance. Section () covers the determinants of the capital structure of corporations in detail.

Additionally, \textcite{ledgerwood2006transforming} attribute the capital structure of MFIs to maturity and the institutional life cycle. The institutional life cycle theory posits that the financing stricture of an MFI varies with the stage of the MFI in the life cycle. At the early stages, the MFIs finance their operations mainly through donations and concessionary funds, given that commercial funders find MFIs too risky. At the early stage, most MFIs operate as NGOs. In the second phase, MFIs tend to focus on the expansion of operations and thus look beyond donations to meet their financing needs. The MFIs supplement their donor funding and subsidies with equity financing (from NGOs and public investors) and seed capital from international finance institutions. At the final consolidation phase, MFIs concentrate on attaining sustainability and formalizing their operations. At the consolidation phase, MFIs raise additional capital through debt (using foreign donor funds as guarantees) and deposits \textcite{bayai2016financing}.

Researchers like \textcite{bayai2016financing} attribute part of the disparities to regulatory provisions relating to the ways MFIs can raise capital as well as historical legacies on saving and lending. Thus, the level of maturity of an MFI is likely to affect the capital structure of MFIs. For individual sources of capital, research indicates that donors are likely to contribute more to MFIs that have verifiable accounting information, and those MFIs that emphasize on their social mission \textcite{hudon2010management}.

Also, donors funding is positively correlated with the tangible assets, past due loans and the size of an MFI, indicating that even donors evaluate the riskiness of the MFIs they fund, using this information as a screening mechanism. However, donor funding is not influenced by the profitability of the MFI, meaning that indeed donors mostly have a social orientation. On the other hand, more profitable firms (with more even profit distribution) with more tangible assets (and bigger) and with a corporate organizational structure have easier access to debt capital. Moreover, the credit ratings of the MFIs do not influence their capital structure \autocite{tchuigoua2014institutional}.


%%%%% REFERENCES

% JEM: Quote for the top of references (just like a chapter quote if you're using them).  Comment to skip.
% \begin{savequote}[8cm]
% The first kind of intellectual and artistic personality belongs to the hedgehogs, the second to the foxes \dots
%   \qauthor{--- Sir Isaiah Berlin \cite{berlin_hedgehog_2013}}
% \end{savequote}

\setlength{\baselineskip}{0pt} % JEM: Single-space References

{\renewcommand*\MakeUppercase[1]{#1}%
\printbibliography[heading=bibintoc,title={\bibtitle}]}

\end{document}
